Internet of Things comprises an increasing number of interconnected smart devices, where communication happens anytime, anywhere, reducing hardware costs and the complexity of the architectures. Reading Radio Frequency Identification tags using Ultra High Frequency is an activity that can generate a large amount of data due to the tag reader of that frequency. This work proposes an architecture that implements the reading of Radio Frequency Identification tags using Ultra High Frequency with a low cost tag reader in relation to those that use this frequency in an infrastructure with cloud computing and microservices. The use of cloud computing and microservices is necessary because of the scalability and flexibility for the large volume of data that can be generated in the reading of Radio Frequency Identification tags using Ultra High Frequency. The proposed architecture was applied in a real case study to verify their adherence and compliance, in addition to showing up properly performed the case study. The results obtained demonstrate that the tag reading board with Radio Frequency Identification chosen obtained a satisfactory performance in the reading of tags, as well as the architectural decisions proposed in the work. In scenarios where reading distance is a fundamental requirement, it is necessary to include an external antenna for better tag reading results. The use of cloud computing and microservices have been shown to have a high development cost in the proposed architecture due to its complexities and the amount of resources created to implement the architecture.