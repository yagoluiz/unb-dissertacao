\label{apendice}

Abaixo é apresentado os repositórios onde estão localizados todos os códigos fontes implementados neste trabalho:

\begin{itemize}
    \item \textbf{\textit{Middleware} \acrshort{IoT}:} \url{https://github.com/yagoluiz/rfid-reader-middleware}
    \item \textbf{Microsserviços e serviço \textit{Serverless}:} \url{https://github.com/yagoluiz/rfid-reader-api}
    \item \textbf{Aplicação \textit{front-end}:} \url{https://github.com/yagoluiz/rfid-reader-web}
\end{itemize}

\section{Monitoramento da telemetria realizado pela placa RFID}

\begin{lstlisting}[language={Java}, label=telemetria-placa]
public class DeviceMessageSender implements Runnable {

    @Override
    public void run() {
        try {
            // Telemetria realizada de modo continuo
            while (true) {
                // Informacoes de telemetria:
                // (IP, temperatura e status da conexao)
                TelemetryModel telemetryModel = new TelemetryModel(
                    InetAddress.getLocalHost().getHostAddress(),
                    ReaderService.getTemperature(),
                    TelemetryTypeEnum.TELEMETRY.getEnumValue(),
                    ReaderService.isConnection());

                // Informacoes de telemetria no formato JSON
                String telemetryJson = new JsonUtil()
                    .serialize(telemetryModel);
                Message msg = new Message(telemetryJson);

                // Informacoes de callback para biblioteca Iot Hub
                Object lock = new Object();
                DeviceEventCallback callback = new DeviceEventCallback();

                // Envio das informacoes de telemetria:
                // (JSON com informacoes e callback)
                IoTHubService.sendEventDeviceAsync(msg, callback, lock);

                // Sincronizacao das informacoes
                synchronized (lock) {
                    lock.wait();
                }

                System.out.println("Sending message Telemetry: "
                    + telemetryJson);

                // Telemetria realizada a cada 1 segundo
                Thread.sleep(1000);
            }
        } catch (InterruptedException | UnknownHostException |
            URISyntaxException | ReaderException e) {
            System.out.println("Interrupted telemetry");
            e.printStackTrace();
        }
    }

    // Implementacao da biblioteca IoT Hub para envio de informacoes
    public static class DeviceEventCallback implements IotHubEventCallback {
        public void execute(IotHubStatusCode status, Object context) {
            System.out.println("IoT Hub responded to message with status: "
                + status.name());

            if (context != null) {
                synchronized (context) {
                    context.notify(); // Sincronizacao de informacoes
                }
            }
        }
    }
}
\end{lstlisting}

\section{Leitura e log realizado pela placa RFID}

\begin{lstlisting}[language={Java}, label=leitura-log-placa]
public class ReaderMessageSender implements Runnable {

    @Override
    public void run() {
        try {
            TagReadData[] tagReads;
            // Leitura realizada de modo continuo
            while (true) {
                // Leitura realizada a cada 1 segundo
                Thread.sleep(1000);
                // Limite de 10 tags lidas a cada 1 milissegundo
                tagReads = ReaderService.readDataTag(10);
                for (TagReadData tr : tagReads) {
                    if (!tr.epcString().isEmpty()) {
                    
                        // Informacoes de leitura:
                        // (IP, EPC, data e antena)
                        ReadModel readModel = new ReadModel(
                            InetAddress.getLocalHost().getHostAddress(),
                            tr.epcString(),
                            LocalDateTime.now().toString(),
                            tr.getAntenna(),
                            TelemetryTypeEnum.READ.getEnumValue());

                        // Informacoes de leitura no formato JSON
                        String readJson = new JsonUtil()
                            .serialize(readModel);
                        Message msg = new Message(readJson);

                        // Informacoes de callback para biblioteca Iot Hub
                        Object lock = new Object();
                        DeviceMessageSender.DeviceEventCallback callback =
                            new DeviceMessageSender.DeviceEventCallback();

                        // Envio das informacoes de leitura:
                        //(JSON com informacoes e callback)
                        IoTHubService
                            .sendEventDeviceAsync(msg, callback, lock);

                        // Sincronizacao das informacoes
                        synchronized (lock) {
                            lock.wait();
                        }

                        System.out.println("Sending message Read: "
                            + readJson);
                    }
                }
            }
        } catch (Exception e) {
            try {
                // Informacoes de erro na leitura da placa RFID
                synchronizeException(e);
            } catch (URISyntaxException | UnknownHostException se) {
                se.printStackTrace();
            }
        }
    }


    private static void synchronizeException(Exception exception)
        throws URISyntaxException, UnknownHostException {
        // Informacoes de erro:
        // (IP, nome do erro, pilha de erros e data)
        LogModel logModel = new LogModel(
                InetAddress.getLocalHost().getHostAddress(),
                exception.getMessage(),
                exception.getStackTrace().toString(),
                LocalDateTime.now().toString(),
                TelemetryTypeEnum.LOG.getEnumValue());

        // Informacoes de erro no formato JSON
        String logJson = new JsonUtil().serialize(logModel);
        Message msg = new Message(logJson);

        // Informacoes de callback para biblioteca Iot Hub
        Object lock = new Object();
        DeviceMessageSender.DeviceEventCallback callback = new DeviceMessageSender.DeviceEventCallback();

        // Envio das informacoes de erro (JSON com informacoes e callback)
        IoTHubService.sendEventDeviceAsync(msg, callback, lock);

        System.out.println("Sending message Log: " + logJson);
    }
}
\end{lstlisting}

\section{\textit{Azure Functions} com informações providas do \textit{IoT Hub}}

\begin{lstlisting}[language={[Sharp]C}, label=azure-functions]
public static class IoTTriggerFunction
{
        [FunctionName("IoTTriggerFunction")]
        public static async Task Run(
            // Configuracao de escuta de mensagem via IoT Hub
            [IoTHubTrigger("messages/events",
            Connection = "IoTHubConnectionString",
            ConsumerGroup = "$Default")]EventData[] events,
            // Configuracao de banco de dados Cosmos DB para evento de telemetria
            [CosmosDB(databaseName: "Rfid", collectionName: "Telemetry",
            ConnectionStringSetting = "CosmosDBConnection")]
            DocumentClient clientTelemetry,
            // Configuracao de banco de dados Cosmos DB para evento de leitura
            [CosmosDB(databaseName: "Rfid", collectionName: "Read",
            ConnectionStringSetting = "CosmosDBConnection")]
            DocumentClient clientRead,
            // Configuracao de Blob Storage para evento de log
            [Blob("logs", Connection = "BlobConnectionString")]
            CloudBlobContainer blobContainer,
            ILogger log)
        {
            log.LogInformation
                ($"EventHubTriggerFunction executed: {DateTime.Now}");

            // Listagem de mensagens do IoT Hub
            foreach (EventData eventData in events)
            {
                // Informacoes da mensagem
                var messageBody = Encoding.UTF8
                    .GetString(eventData.Body.Array,
                    eventData.Body.Offset,
                    eventData.Body.Count);
                    
                // Informacao do tipo de mensagem:
                // (Telemetria, leitura ou log)
                var telemetryTypeModel = JsonConvert
                    .DeserializeObject<TelemetryTypeModel>(messageBody);

                log.LogInformation
                    ($"EventHubTriggerFunction message: {messageBody}");
                log.LogInformation
                    ($"EventHubTriggerFunction telemetry type:
                        {telemetryTypeModel.TelemetryType}");

                // Envio de mensagem de acordo com o tipo
                switch (telemetryTypeModel.TelemetryType)
                {
                    // Informacoes de telemetria (Cosmos BD)
                    case (int)TelemetryTypeEnum.TELEMETRY:
                        // Informacoes de telemetria
                        var telemetryModel = JsonConvert
                            .DeserializeObject<TelemetryModel>
                            (messageBody);
                        // Envio das informacoes de telemetria
                        await clientTelemetry.CreateDocumentAsync
                            (UriFactory.CreateDocumentCollectionUri
                            ("Rfid", "Telemetry"), telemetryModel);
                        break;
                    // Informacoes de leitura (Cosmos BD)
                    case (int)TelemetryTypeEnum.READ:
                        // Informacoes de leitura
                        var readModel = JsonConvert
                            .DeserializeObject<ReadModel>(messageBody);
                        // Envio das informacoes de leitura
                        await clientRead.CreateDocumentAsync
                            (UriFactory.CreateDocumentCollectionUri
                            ("Rfid", "Read"), readModel);
                        break;
                    // Informacoes de log (Blob Storage)
                    case (int)TelemetryTypeEnum.LOG:
                        // Informacoes de log
                        var logName = Guid.NewGuid().ToString();
                        var cloudBlockBlob = blobContainer
                            .GetBlockBlobReference($"{logName}.json");
                        // Envio das informacoes de log
                        await cloudBlockBlob.UploadTextAsync(messageBody);
                        break;
                        
                    default:
                        log.LogInformation
                        ($"EventHubTriggerFunction telemetry type invalid: {telemetryTypeModel.TelemetryType}");
                        break;
                }

                // Lista proxima mensagem apos o processamento da anterior
                await Task.Yield();
            }

            log.LogInformation
                ($"EventHubTriggerFunction finished: {DateTime.Now}");
        }
}
\end{lstlisting}

\section{Configuração de autenticação e autorização do cliente na \textit{API Gateway}}

\begin{lstlisting}[language={[Sharp]C}, label=gateway-configuracao]
public class Startup
{
    // Configuracao de variaveis de ambiente
    public Startup(IConfiguration configuration)
    {
        Configuration = configuration;
    }

    // Variaveis de ambiente
    public IConfiguration Configuration { get; }

    public void ConfigureServices(IServiceCollection services)
    {
        // Configuracao de autenticacao de cliente do API Gateway
        void options(IdentityServerAuthenticationOptions identityServer)
        {
            identityServer.Authority = Configuration
                .GetSection("IdentityServer:Authority").Value;
            identityServer.ApiName = Configuration
                .GetSection("IdentityServer:ApiName").Value;
            identityServer.ApiSecret = Configuration
                .GetSection("IdentityServer:ApiSecret").Value;
            identityServer.SupportedTokens = SupportedTokens.Both;
        }

        // Configuracao de versao do framework ASP.NET Core
        services.AddMvc()
            .SetCompatibilityVersion(CompatibilityVersion.Version_2_2);
        // Configuracao de CORS
        services.AddCors();
        // Configuracao de biblioteca Ocelot
        services.AddOcelot(Configuration);
        // Configuracao de autenticacao de cliente da API Gateway
        services.AddAuthentication()
        .AddIdentityServerAuthentication
            (Configuration.GetSection("IdentityServer:ProviderKey")
                .Value, options);
    }

    public void Configure(IApplicationBuilder app, IHostingEnvironment env)
    {
        // Configuracao de ambiente de desenvolvimento e producao
        if (env.IsDevelopment())
        {
            app.UseDeveloperExceptionPage();
        }
        else
        {
            app.UseHsts();
        }

        // Configuracao utilizacao de HTTPS
        app.UseHttpsRedirection();
        // Configuracao de CORS para acesso aos endpoints
        app.UseCors(x =>
        {
            x.AllowAnyOrigin();
            x.AllowAnyHeader();
            x.AllowAnyMethod();
            x.AllowCredentials();
        });
        // Configuracao de biblioteca Ocelot
        app.UseOcelot().Wait();
    }
}
\end{lstlisting}

\section{\textit{Endpoints} disponibilizados pela \textit{API Gateway}}

\begin{lstlisting}[language=Java, label=gateway-endpoints]
{
  // Configuracao de endpoints
  "ReRoutes": [
    {
      "DownstreamPathTemplate": "/connect/token", // Endpoint microsservico de Identidade
      "DownstreamScheme": "https", // Configuracao de HTTPS
      "DownstreamHostAndPorts": [
        {
          "Host": "sso.api", // Local de hospedagem
          "Port": 443 // Porta TCP
        }
      ],
      "UpstreamPathTemplate": "/connect/token", // Endpoint API Gateway
      "UpstreamHttpMethod": [ "Post" ] // Verbo HTTP
    },
    {
      "DownstreamPathTemplate": "/api/{version}/assets/read", // Endpoint microsservico de Ativo
      "DownstreamScheme": "https", // Configuracao de HTTPS
      "DownstreamHostAndPorts": [
        {
          "Host": "asset.api", // Local de hospedagem
          "Port": 443 // Porta TCP
        }
      ],
      "UpstreamPathTemplate": "/api/{version}/assets/read", // Endpoint API Gateway
      "UpstreamHttpMethod": [ "Get" ], // Verbo HTTP
      "AuthenticationOptions": {
        "AuthenticationProviderKey": "api_gateway", // Permissao de autenticacao do cliente
        "AllowedScopes": [ "gateway" ] // Permissao de scopo
      }
    },
    {
      "DownstreamPathTemplate": "/api/{version}/logs/limit/{limit}", // Endpoint microsservico de Log
      "DownstreamScheme": "https", // Configuracao de HTTPS
      "DownstreamHostAndPorts": [
        {
          "Host": "log.api", // Local de hospedagem
          "Port": 443 // Porta TCP
        }
      ],
      "UpstreamPathTemplate": "/api/{version}/logs/limit/{limit}", // Endpoint API Gateway
      "UpstreamHttpMethod": [ "Get" ], // Verbo HTTP
      "AuthenticationOptions": {
        "AuthenticationProviderKey": "api_gateway", // Permissao de autenticacao do cliente
        "AllowedScopes": [ "gateway" ] // Permissao de scopo
      }
    },
    {
      "DownstreamPathTemplate": "/api/{version}/reads/limit/{limit}", // Endpoint microsservico de Leitura
      "DownstreamScheme": "https", // Configuracao de HTTPS
      "DownstreamHostAndPorts": [
        {
          "Host": "read.api", // Local de hospedagem
          "Port": 443 // Porta TCP
        }
      ],
      "UpstreamPathTemplate": "/api/{version}/reads/limit/{limit}", // Endpoint API Gateway
      "UpstreamHttpMethod": [ "Get" ], // Verbo HTTP
      "AuthenticationOptions": {
        "AuthenticationProviderKey": "api_gateway", // Permissao de autenticacao do cliente
        "AllowedScopes": [ "gateway" ] // Permissao de scopo
      }
    },
    {
      "DownstreamPathTemplate": "/api/{version}/telemetries/limit/{limit}", // Endpoint microsservico de Telemetria
      "DownstreamScheme": "https", // Configuracao de HTTPS
      "DownstreamHostAndPorts": [
        {
          "Host": "telemetry.api", // Local de hospedagem
          "Port": 443 // Porta TCP
        }
      ],
      "UpstreamPathTemplate": "/api/{version}/telemetries/limit/{limit}", // Endpoint API Gateway
      "UpstreamHttpMethod": [ "Get" ], // Verbo HTTP
      "AuthenticationOptions": {
        "AuthenticationProviderKey": "api_gateway", // Permissao de autenticacao do cliente
        "AllowedScopes": [ "gateway" ] // Permissao de scopo
      }
    }
  ],
  "GlobalConfiguration": {}
}
\end{lstlisting}

\section{Configuração da imagem \textit{Docker} do microsserviço de Ativo}

\begin{lstlisting}[language=docker, label=docker-servicos]
# Imagem do framework ASP.NET Core
FROM mcr.microsoft.com/dotnet/core/aspnet:2.2-stretch-slim AS base
WORKDIR /app
EXPOSE 80
EXPOSE 443

# Restauracao e compilacao do projeto API
FROM mcr.microsoft.com/dotnet/core/sdk:2.2-stretch AS build
WORKDIR /src
COPY . .
WORKDIR /src/services/asset/Asset.API
RUN dotnet restore
RUN dotnet build --no-restore -c Release -o /app

# Publicacao do projeto compilado
FROM build AS publish
RUN dotnet publish --no-restore -c Release -o /app

# Publicacao da imagem
FROM base AS final
WORKDIR /app
COPY --from=publish /app .
ENTRYPOINT ["dotnet", "Asset.API.dll"]
\end{lstlisting}

\section{Configuração da imagem \textit{Docker} da aplicação \textit{back-end}}

\begin{lstlisting}[language=docker, label=docker-web]
# Imagem do framework Node.js
FROM node:latest as node

# Execucao de comandos para instalacao
WORKDIR /app
COPY . .
RUN npm install
RUN npm install node-sass
RUN npm run build --prod

# Publicacao da imagem
FROM nginx:alpine
COPY --from=node /app/dist/rfid-reader-web /usr/share/nginx/html
\end{lstlisting}

\section{Configuração das imagens \textit{Docker} dos microsserviços e da \textit{API Gateway} no arquivo \textit{Docker Compose}}

\begin{lstlisting}[language=docker-compose-2, label=ci-docker-servicos]
version: '3.4'

# Configuracao das imagens dos microsservicos e da API Gateway
services:
# Configuracao da imagem da API Gateway
  ocelot.api:
    image: ${DOCKER_REGISTRY-}ocelotapi
    build:
      context: .
      dockerfile: gateway/ocelot/Ocelot.API/Dockerfile
# Configuracao da imagem do microsservico de Identidade
  sso.api:
    image: ${DOCKER_REGISTRY-}ssoapi
    build:
      context: .
      dockerfile: security/sso/SSO.API/Dockerfile
# Configuracao da imagem do microsservico de Ativo
  asset.api:
    image: ${DOCKER_REGISTRY-}assetapi
    build:
      context: .
      dockerfile: services/asset/Asset.API/Dockerfile
# Configuracao da imagem do microsservico de Log
  log.api:
    image: ${DOCKER_REGISTRY-}logapi
    build:
      context: .
      dockerfile: services/log/Log.API/Dockerfile
# Configuracao da imagem do microsservico de Leitura
  read.api:
    image: ${DOCKER_REGISTRY-}readapi
    build:
      context: .
      dockerfile: services/read/Read.API/Dockerfile
# Configuracao da imagem do microsservico de Telemetria
  telemetry.api:
    image: ${DOCKER_REGISTRY-}telemetryapi
    build:
      context: .
      dockerfile: services/telemetry/Telemetry.API/Dockerfile
\end{lstlisting}

\section{Configuração de integração contínua dos microsserviços e serviços \textit{Serverless}}

\begin{lstlisting}[language=docker-compose, label=ci-yaml-servicos]
# Branch na qual a integracao continua sera inicializada
trigger:
  branches:
    include:
    - master

# Maquina virtual onde sera realizada a integracao continua
pool:
  vmImage: 'ubuntu-latest'

# Repositorio atual do arquivo yaml
resources:
- repo: self

# Variaveis de ambiente para a etapa de steps
variables:
  buildConfiguration: 'Release'
  restoreBuildProjectFunction: '**/IoT.Function.Trigger.csproj'
  restoreBuildProjectFunctionTest: '**/IoT.Function.Trigger.*[Tt]ests/*.csproj'
  azureSubscriptionEndpoint: azure-resource
  azureContainerRegistry: rfidregistry.azurecr.io

# Etapa de complicacao e restauracao dos projetos (Serverless e containers)
steps:
# Restauracao do projeto Azure Function
- task: DotNetCoreCLI@2
  displayName: Restore
  inputs:
    command: restore
    projects: '$(restoreBuildProjectFunction)'

# Compilacao do projeto Azure Function
- task: DotNetCoreCLI@2
  displayName: Build
  inputs:
    projects: '$(restoreBuildProjectFunction)'
    arguments: '--configuration $(BuildConfiguration)'

# Execucao de testes do projeto Azure Function
- task: DotNetCoreCLI@2
  displayName: Test
  inputs:
    command: test
    projects: '$(restoreBuildProjectFunctionTest)'
    arguments: '--configuration $(buildConfiguration) --collect "Code coverage"'

# Compilacao para arquivo ZIP do projeto Azure Function
- task: DotNetCoreCLI@2
  displayName: Publish
  inputs:
    command: publish
    arguments: '--configuration $(BuildConfiguration) --output $(Build.ArtifactStagingDirectory)'
    projects: '$(restoreBuildProjectFunction)'
    publishWebProjects: false
    modifyOutputPath: true
    zipAfterPublish: true

# Publicacao no arquivo ZIP do projeto Azure Function
- task: PublishBuildArtifacts@1
  displayName: 'Publish Artifact'
  inputs:
    pathtoPublish: '$(Build.ArtifactStagingDirectory)'

# Compilacao dos projetos em containers
- task: DockerCompose@0
  displayName: Build services
  inputs:
    action: Build services
    azureSubscriptionEndpoint: $(azureSubscriptionEndpoint)
    azureContainerRegistry: $(azureContainerRegistry)
    dockerComposeFile: '**/docker-compose.yml'
    projectName: $(Build.Repository.Name)
    qualifyImageNames: true
    includeLatestTag: true
    additionalImageTags: $(Build.BuildId)

# Publicacao dos projetos em containers
- task: DockerCompose@0
  displayName: Push services
  inputs:
    action: Push services
    azureSubscriptionEndpoint: $(azureSubscriptionEndpoint)
    azureContainerRegistry: $(azureContainerRegistry)
    dockerComposeFile: '**/docker-compose.yml'
    projectName: $(Build.Repository.Name)
    qualifyImageNames: true
    includeLatestTag: true
    additionalImageTags: $(Build.BuildId)
\end{lstlisting}

\section{Configuração de integração contínua da aplicação \textit{front-end}}

\begin{lstlisting}[language=docker-compose, label=ci-yaml-web]
# Branch na qual a integracao continua sera inicializada
trigger:
  branches:
    include:
    - master

# Maquina virtual onde sera realizada a integracao continua
pool:
  vmImage: 'ubuntu-latest'

# Repositorio atual do arquivo yaml
resources:
- repo: self

# Variaveis de ambiente para a etapa de steps
variables:
  azureSubscriptionEndpoint: azure-resource
  azureContainerRegistry: rfidregistry.azurecr.io
  imageName: rfidreaderweb

# Etapa de complicacao e restauracao do projeto em container
steps:
# Compilacao do projeto em container
- task: Docker@1
  displayName: 'Build an image'
  inputs:
    command: build
    azureSubscriptionEndpoint: $(azureSubscriptionEndpoint)
    azureContainerRegistry: $(azureContainerRegistry)
    imageName: $(imageName)
    includeLatestTag: true

# Publicacao do projeto em container
- task: Docker@1
  displayName: 'Push an image'
  inputs:
    command: push
    azureSubscriptionEndpoint: $(azureSubscriptionEndpoint)
    azureContainerRegistry: $(azureContainerRegistry)
    imageName: $(imageName)
    includeLatestTag: true
\end{lstlisting}