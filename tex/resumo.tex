Internet das Coisas envolve um número crescente de dispositivos inteligentes interconectados em que sua comunicação acontece a qualquer hora e em qualquer lugar, sendo possível reduzir custos de \textit{hardware} de arquiteturas complexas. A leitura de tags com Identificação por Radiofrequência utilizando Ultra Alta Frequência é uma atividade que pode gerar um grande volume de dados, devido ao leitor de tags dessa frequência. Este trabalho propõe uma arquitetura que implementa a leitura de tags com Identificação por Radiofrequência utilizando Ultra Alta Frequência com um leitor de tags de baixo custo de mercado em relação aos que utilizam essa frequência em uma infraestrutura com computação em nuvem e microsserviços. A utilização da computação em nuvem e microsserviços se fazem necessários devido à escalabilidade e flexibilidade para o grande volume de dados que podem ser gerados na leitura de tags com Identificação por Radiofrequência que utilizam Ultra Alta Frequência. A arquitetura proposta foi aplicada em um estudo de caso real para verificar a sua aderência e conformidade, além de mostrar-se adequada ao estudo de caso realizado. Os resultados obtidos demonstraram que a placa de leitura de tags com Identificação por Radiofrequência escolhida obteve desempenho satisfatório na leitura de tags, assim como as decisões arquiteturais propostas no trabalho. Em cenários onde a distância de leitura é um requisito fundamental, é necessário incluir uma antena externa para obter melhores resultados de leitura de tags. A utilização de computação em nuvem e microsserviços demonstraram ter um custo alto de desenvolvimento na arquitetura proposta, devido às suas complexidades e à quantidade de recursos criados para implementação da arquitetura.